\beginsong{10 w skali Beauforta}[by={sł. Janusz Kondratowicz; muz. Krzysztof Klenczon;\break wyk. Trzy Korony}]
\calcchordswidth{\[F G7 C E a]}
\beginverse
\clist{\[a d]}Kołysał nas zachodni wiatr,
\clist{\[E a]}A brzeg gdzieś za rufą został.
\clist{\[d a]}I nagle ktoś jak papier zbladł,
\clist{\[d E]}Sztorm idzie panie bosman.
\endverse
\beginchorus\memorize[chorus]
\clist{\[F C F C]}A bosman tylko zapiął płaszcz
\clist{\[F E a]}I zaklął ,,ech do czorta''.
\clist{\[F G7 C E a]}Nie daje łajbie żadnych szans
\clist{\[d E a]}10 w skali Beauforta.
\endchorus
\beginverse
\clist{^}Z zasłony ołowianych chmur,
\clist{^}Ulewa spadła nagle.
\clist{^}Rzucało nami w górę, w dół,
\clist{^}A fala zmyła żagle.
\endverse
\ifphone
\beginchorus\replay[chorus]
\clist{^}A bosman tylko zapiął płaszcz
\clist{^}I zaklął ,,ech do czorta''.
\clist{^}Nie daje łajbie żadnych szans
\clist{^}10 w skali Beauforta.
\endchorus
\fi
\beginverse
% zwrotka z pierwszej wersji
\clist{^}~~Gdzie został ciepły, cichy kąt
\clist{^}~~I brzegu kształt znajomy
\clist{^}~~Zasnuły mgły daleki ląd
\clist{^}~~Dokładnie, z każdej strony.
\endverse
\ifphone
\beginchorus\replay[chorus]
\clist{^}A bosman tylko zapiął płaszcz
\clist{^}I zaklął ,,ech do czorta''.
\clist{^}Nie daje łajbie żadnych szans
\clist{^}10 w skali Beauforta.
\endchorus
\fi
\beginverse
\clist{^}O pokład znów uderzył deszcz
\clist{^}I padał już do rana.
\clist{^}Cholernie ciężki był to rejs,
\clist{^}Szczególnie dla bosmana.
\endverse
\beginchorus\replay[chorus]
\clist{^}A bosman tylko zapiął płaszcz
\clist{^}I zaklął ,,ech do czorta''.
\clist{^}Przedziwne czasem sny się ma,
\clist{^}10 w skali Beauforta.
\endchorus
\ifbosman
\vspace{4cm}\break
\textnote{Wersja bosmańska (niecenzuralna)\hfill\break -- autor słów nieznany:}
\beginverse
\clist{^}Kołysał nas zachodni wiatr,
\clist{^}Brzeg był za rufą z dala
\clist{^}I nagle ktoś jak papier zbladł:
\clist{^},,Grot nam się rozpierdala!''
\endverse
\beginchorus\replay[chorus]
\clist{^}A bosman tylko zapiął płaszcz
\clist{^}I zaklął: ,,Mać jebana!
\clist{^}Nie mogłaś szmato, kurwa mać,
\clist{^}Zaczekać z tym do rana!!?''
\endchorus
\beginverse
\clist{^}Spod ciemnych ołowianych chmur
\clist{^}Ulewa spadła nagle,
\clist{^}A myśmy, jak te chuje dwa,
\clist{^}W kubryku szyli żagle.
\endverse
\beginchorus\replay[chorus]
\clist{^}A bosman do kubryku wpadł
\clist{^}I zaklął: ,,Chuj wam w dupę!
\clist{^}Złamali znowu igły dwie,
\clist{^}Gdzie ja dziś takie kupię?''
\endchorus
\beginverse
\clist{^}W nawigacyjnej Stary siadł,
\clist{^}Słuchając komunikat.
\clist{^}Wkurwiony maksymalnie klął:
\clist{^},,O żesz mać, znowu dycha!''
% Wkurzony maksymalnie rzekł:
% "Panowie, znowu dycha!" (10B)
\endverse
\beginchorus\replay[chorus]
\clist{^}A bosman tylko w mesie siadł
\clist{^}Ponury jak na stypie.
\clist{^},,Jak dalej będzie piździć tak,
\clist{^}Rozpieprzy nam tę krypę!''
\endchorus
\beginverse
\clist{^}W kambuzie kuk, rzygając w krąg,
\clist{^}Ponuro żuł nienawiść:
\clist{^},,Że też te gnoje jeszcze żrą,
\clist{^}Toż takich można zabić!''
\endverse
\beginchorus\replay[chorus]
\clist{^}A bosman tylko w mesie siadł
\clist{^}I oparł nos na blacie:
\clist{^},,Obiad ma być dziś z trzech dań
\clist{^}I deser na dodatek!''
\endchorus
\beginverse
\clist{^}Kompasu igła całą noc
\clist{^}Tańczyła rock and rolla,
\clist{^}A wściekły sternik czuł, że go
\clist{^}Ogarnia paranoja...
\endverse   
\beginchorus    \replay[chorus]
\clist{^}A bosman do kokpitu wpadł
\clist{^}I ujął szturwał w ręce:
\clist{^},,Do kurwy nędzy, równo jedź,
\clist{^}Bo jaja ci ukręcę!''
\endchorus  
\beginverse 
\clist{^}Gdy słońce wyszło spoza chmur,
\clist{^}A wicher się wyszalał,
\clist{^}To bosman tylko flachę wziął
\clist{^}I szybko pałę zalał.
\endverse
\beginchorus\replay[chorus]
\clist{^}A potem tylko zapiął płaszcz,
\clist{^}Chciał zakląć -- lecz nie zaklął,
\clist{^}Odchamić się najwyższy czas,
\clist{^}Więc tylko ręką machnął.
\endchorus
\beginverse
\clist{^}Gdy słońce zgasło i gdy sztorm
\clist{^}Wydmuchał się do woli,
\clist{^}W bosmańskiej brodzie zakwitł blask
\clist{^}Setek kryształków soli.
\endverse
\beginchorus\replay[chorus]
\clist{^}A bosman tylko zapiął płaszcz
\clist{^}I zaklął... otóż nie zaklął!
\clist{^}Bosman znów zaczął mówić nam
\clist{^}Piękną poprawną polszczyzną!
\endchorus
\fi
\endsong
\ifbosman
\beginscripture{}
Zwrotka ,,Gdzie został ciepły, cichy kąt...'' jest z pierwotnej wersji utworu.
\endscripture
\fi
